\documentclass[12pt,onecolumn]{article}
\usepackage{amsmath}
\usepackage{epsfig}
\usepackage{fullpage}
\usepackage{xcolor}
\usepackage{amssymb}
\usepackage{fancyhdr}

\newcommand{\JJ}{{\bf J}}
\newcommand{\vv}{{\bf v}}
\newcommand{\uu}{{\bf u}}
\newcommand{\II}{{\bf I}}
\newcommand{\BB}{{\bf B}}
\newcommand{\FF}{{\bf F}}
\newcommand{\RR}{{\bf R}}
\newcommand{\UU}{{\bf U}}
\newcommand{\qq}{{\bf q}}
\newcommand{\pp}{{\bf p}}
\newcommand{\VV}{{\bf V}}
\newcommand{\DD}{{\bf D}}
\newcommand{\TT}{{\bf T}}
\newcommand{\trans}{{\bf t}}
\newcommand{\ff}{{\bf f}}
\newcommand{\xx}{{\bf x}}
\newcommand{\boldt}{{\bf t}}
\newcommand{\bolda}{{\bf a}}
\newcommand{\boldb}{{\bf b}}
\newcommand{\bb}{{\bf b}}
\newcommand{\fancyS}{{\mathcal S}}
\newcommand{\boldS}{{\bf S}}
\newcommand{\boldA}{{\bf A}}
\newcommand{\rr}{{\bf r}}
\newcommand{\cc}{{\bf c}}
\newcommand{\dd}{{\bf d}}
\newcommand{\QQ}{{\bf Q}}
\newcommand{\MM}{{\bf M}}
\newcommand{\HH}{\bf H}
\newcommand{\boldB}{{\bf B}}
\newcommand{\CC}{{\bf C}}
\newcommand{\Top}{{\mathbb T}}
\newcommand{\Bottom}{{\mathbb B}}
\newcommand{\functionf}{{\mathbb F}}
\newcommand{\functionr}{{\mathbb R}}
\newcommand{\functions}{{\mathbb S}}
\newcommand{\DB}{\mathbb{D}\Bottom}
\DeclareMathOperator{\sech}{sech}
\DeclareMathOperator{\curl}{curl}

\fancyhf{}
\lfoot{ \fancyplain{}{Page \thepage} }
\rfoot{ \fancyplain{}{\today} }
\pagestyle{fancyplain}

\begin{document}
\vspace*{1em}
\section{Working Notes on Laplacian Eigenfunctions}
%Revision: \today
\subsection{Analytic Solutions}

One vorticity basis function is:
\begin{equation}
\phi_a = \sin(a_1x) \sin(a_2 y) \bolda_z.
\end{equation}
Its curl is:
\begin{equation}
\curl \phi_a = a_2 \sin(a_1x) \cos(a_2 y) \bolda_x - a_1 \cos(a_1 x) \sin(a_2 y) \bolda_y.
\end{equation}
Crossed with another basis function, we get:
\begin{eqnarray}
\curl \phi_a \times \curl \phi_b &=& a_1 b_2 \cos(a_1 x) \cos(b_2 y) \sin(a_2 x) \sin(b_1 y) - \\
&& a_2 b_1 \cos(a_2 x) \cos(b_1 y) \sin(a_1 x) \sin(b_2 y).
\end{eqnarray}
{\bf This is a typo in the DeWitt thesis. The actual function is:}
\begin{eqnarray}
\curl \phi_a \times \curl \phi_b &=& a_1 b_2 \cos(a_1 x) \cos(b_2 y) \sin(a_2 y) \sin(b_1 x) - \\
&& a_2 b_1 \cos(a_2 y) \cos(b_1 x) \sin(a_1 x) \sin(b_2 y).
\end{eqnarray}
Finally, we are interested in the projection onto a single basis function, $k_1 = 1, k_2 = 1$.
\begin{equation}
C = \int_0^\pi \int_0^\pi (\curl \phi_a \times \curl \phi_b) \cdot \phi_k \;dx\;dy
\end{equation}
The known structure coefficient for a single pair:
\begin{eqnarray}
a_1 &=& 2 \\
a_2 &=& 3 \\
b_1 &=& 1 \\
b_2 &=& 2 \\
k_1 &=& 1 \\
k_2 &=& 1
\end{eqnarray}
is $-0.05 = -\frac{1}{20}$. According to the thesis, it should work out to $-\frac{1}{4}(a_1 b_2 - a_2 b_1) = -\frac{1}{4}(2 \cdot 2 - 3 \cdot 1) = -\frac{1}{4}$. The final $\frac{1}{5}$ factor should come in when the wave number is finally added at the very end, $\lambda_j^{-1} = \frac{1}{b_1^2 + b_2^2} = \frac{1}{5}$, giving the final coefficient $-\frac{1}{4}\frac{1}{5} = -0.05$.

Plugging into Mathematica for a sanity check. Over $[0,\pi] \times [0, \pi]$, the function integrates to $-\frac{\pi^2}{16}$. Presumably the $\pi^2$ factor comes from the two integrals over $[0,\pi]$. When we instead integrate over $[-\pi, \pi] \times [-\pi, \pi]$, we get the more expected $-\frac{\pi^2}{4}$.

Try for another known pair:

The known structure coefficient for a single pair:
\begin{eqnarray}
a_1 &=& 2 \\
a_2 &=& 4 \\
b_1 &=& 1 \\
b_2 &=& 3 \\
k_1 &=& 1 \\
k_2 &=& 1
\end{eqnarray}
is $-0.05 = -\frac{1}{20}$, and the symmetric pair:
\begin{eqnarray}
a_1 &=& 4 \\
a_2 &=& 2 \\
b_1 &=& 3 \\
b_2 &=& 1 \\
k_1 &=& 1 \\
k_2 &=& 1
\end{eqnarray}
is $0.025 = \frac{1}{40}$.

\newpage
\subsection{Double Curl}

To compute the eigenfunctions numerically, we need a discrete version of the double curl $\nabla \times \nabla \times = (\nabla \times)^2$ operator. For a velocity field $F = (F_x, F_y, F_z)$, this corresponds to:
\begin{eqnarray*}
(\nabla \times)^2 F 
&=&-\left(\frac{\partial^2 F_x}{\partial y^2} + \frac{\partial^2 F_x}{\partial z^2}\right) + \frac{\partial}{\partial x}\left(\frac{\partial F_y}{\partial y} + \frac{\partial F_z}{\partial z}\right) \mathbf{x} \\
&&  -\left(\frac{\partial^2 F_y}{\partial x^2} + \frac{\partial^2 F_y}{\partial z^2}\right) + \frac{\partial}{\partial y}\left(\frac{\partial F_x}{\partial x} + \frac{\partial F_z}{\partial z}\right) \mathbf{y} \\
&&  -\left(\frac{\partial^2 F_z}{\partial x^2} + \frac{\partial^2 F_z}{\partial y^2}\right) + \frac{\partial}{\partial z}\left(\frac{\partial F_x}{\partial x} + \frac{\partial F_y}{\partial y}\right) \mathbf{z} \\
&=&-\left(\frac{\partial^2 F_x}{\partial y^2} + \frac{\partial^2 F_x}{\partial z^2}\right) + \frac{\partial^2 F_y}{\partial x \partial y} + \frac{\partial^2 F_z}{\partial x \partial z} \; \mathbf{x} \\
&&  -\left(\frac{\partial^2 F_y}{\partial x^2} + \frac{\partial^2 F_y}{\partial z^2}\right) + \frac{\partial^2 F_x}{\partial y \partial x} + \frac{\partial^2 F_z}{\partial y \partial z} \; \mathbf{y} \\
&&  -\left(\frac{\partial^2 F_z}{\partial x^2} + \frac{\partial^2 F_z}{\partial y^2}\right) + \frac{\partial^2 F_x}{\partial z \partial x} + \frac{\partial^2 F_y}{\partial z \partial y} \; \mathbf{z}
\end{eqnarray*}

The mixed derivative can be approximated by rotating the stencil:

\begin{equation}
\frac{\partial^2 F}{\partial x\partial y} = \frac{F_{x + 1,y + 1} - F_{x + 1,y - 1} - F_{x - 1,y + 1} + F_{x - 1,y - 1}}{4 \Delta x \Delta y}
\end{equation}

In 2D, we get:
\begin{eqnarray*}
(\nabla \times)^2 F 
&=&-\frac{\partial^2 F_x}{\partial y^2} + \frac{\partial^2 F_y}{\partial x \partial y} \; \mathbf{x} \\
&&  -\frac{\partial^2 F_y}{\partial x^2} + \frac{\partial^2 F_x}{\partial y \partial x} \; \mathbf{y}
\end{eqnarray*}
\end{document}

xy = 4 cos(2x) cos(2 y) sin(3 x) sin(y) - 3 cos(3 x) cos(y) sin(2 x) sin(2 y)
z = sin(x) sin(y)

integrate (4 cos(2x) cos(2 y) sin(3 x) sin(y) - 3 cos(3 x) cos(y) sin(2 x) sin(2 y)) * sin(x) sin(y) dx dy, x = 0..pi, y=0..pi