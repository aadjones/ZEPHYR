\documentclass[12pt,onecolumn]{article}
\usepackage{amsmath}
\usepackage{epsfig}
\usepackage{fullpage}
\usepackage{xcolor}
\usepackage{amssymb}
\usepackage{fancyhdr}

\newcommand{\JJ}{{\bf J}}
\newcommand{\vv}{{\bf v}}
\newcommand{\uu}{{\bf u}}
\newcommand{\II}{{\bf I}}
\newcommand{\BB}{{\bf B}}
\newcommand{\FF}{{\bf F}}
\newcommand{\RR}{{\bf R}}
\newcommand{\UU}{{\bf U}}
\newcommand{\qq}{{\bf q}}
\newcommand{\pp}{{\bf p}}
\newcommand{\VV}{{\bf V}}
\newcommand{\DD}{{\bf D}}
\newcommand{\TT}{{\bf T}}
\newcommand{\trans}{{\bf t}}
\newcommand{\ff}{{\bf f}}
\newcommand{\xx}{{\bf x}}
\newcommand{\boldt}{{\bf t}}
\newcommand{\bolda}{{\bf a}}
\newcommand{\boldb}{{\bf b}}
\newcommand{\bb}{{\bf b}}
\newcommand{\fancyS}{{\mathcal S}}
\newcommand{\boldS}{{\bf S}}
\newcommand{\boldA}{{\bf A}}
\newcommand{\rr}{{\bf r}}
\newcommand{\cc}{{\bf c}}
\newcommand{\dd}{{\bf d}}
\newcommand{\QQ}{{\bf Q}}
\newcommand{\MM}{{\bf M}}
\newcommand{\HH}{\bf H}
\newcommand{\boldB}{{\bf B}}
\newcommand{\CC}{{\bf C}}
\newcommand{\Top}{{\mathbb T}}
\newcommand{\Bottom}{{\mathbb B}}
\newcommand{\functionf}{{\mathbb F}}
\newcommand{\functionr}{{\mathbb R}}
\newcommand{\functions}{{\mathbb S}}
\newcommand{\DB}{\mathbb{D}\Bottom}
\DeclareMathOperator{\sech}{sech}
\DeclareMathOperator{\curl}{curl}
\DeclareMathOperator{\Left}{left}
\DeclareMathOperator{\Right}{right}

\fancyhf{}
\lfoot{ \fancyplain{}{Page \thepage} }
\rfoot{ \fancyplain{}{\today} }
\pagestyle{fancyplain}

\begin{document}
\vspace*{1em}
\section{Working Notes on Laplacian Eigenfunctions}
%Revision: \today
\subsection{Efficiently Computing the Structure Coefficients}

We define two parts of the the quantity that needs to be integrated:
\begin{eqnarray*}
\Left(a_1, a_2, b_1, b_2, k_1, k_2) &=& \cos(a_1 x) \sin(a_2 y) \sin(b_1 x) \cos(b_2 y) \sin(k_1 x) \sin(k_2 y)\\
\Right(a_1, a_2, b_1, b_2, k_1, k_2) &=& \cos(b_1 x) \sin(b_2 y) \sin(a_1 x) \cos(a_2 y) \sin(k_1 x) \sin(k_2 y)
\end{eqnarray*}
The integral for a specific structure coefficient is then:
\begin{equation}
\int_{-\pi}^\pi\int_{-\pi}^\pi \Left(a_1, a_2, b_1, b_2, k_1, k_2) - \Right(a_1, a_2, b_1, b_2, k_1, k_2) dx \; dy
\end{equation}
The coefficient is only non-zero if one of the following is true for $(a_1, b_1, k_1)$:
\begin{eqnarray*}
a_1 + b_1 &=& k_1 \\
a_1 + k_1 &=& b_1 \\
b_1 + k_1 &=& a _1
\end{eqnarray*}
as well as one of the following for $(a_2, b_2, k_2)$:
\begin{eqnarray*}
a_2 + b_2 &=& k_2 \\
a_2 + k_2 &=& b_2 \\
b_2 + k_2 &=& a _2
\end{eqnarray*}
Note however that this is a necessary condition for a non-zero coefficient, not sufficient. The left and right terms can still cancel each other out.

The double integral for the left then evaluates to:
\begin{equation}
\begin{split}
&\int_{-\pi}^\pi\int_{-\pi}^\pi \Left(a_1, a_2, b_1, b_2, k_1, k_2) dx \; dy = \\
&\frac{1}{4}\left( 
\frac{\sin\left((a_2 - b_2 - k_2)\pi\right)}{a_2 - b_2 - k_2}
+ \frac{\sin\left((a_2 + b_2 - k_2)\pi\right)}{a_2 + b_2 - k_2}
- \frac{\sin\left((a_2 - b_2 + k_2)\pi\right)}{a_2 - b_2 + k_2}
- \frac{\sin\left((a_2 + b_2 + k_2)\pi\right)}{a_2 + b_2 + k_2}
\right) \cdot\\
&\left( 
-\frac{\sin\left((a_1 - b_1 - k_1)\pi\right)}{a_1 - b_1 - k_1}
+ \frac{\sin\left((a_1 + b_1 - k_1)\pi\right)}{a_1 + b_1 - k_1}
+ \frac{\sin\left((a_1 - b_1 + k_1)\pi\right)}{a_1 - b_1 + k_1}
- \frac{\sin\left((a_1 + b_1 + k_1)\pi\right)}{a_1 + b_1 + k_1}
\right)
\end{split}
\end{equation}
Plugging the non-zero conditions directly into this causes a divide by zero error. The correct way to compute things is to plug in a $\pi$ into the terms that have been met. For example, for the following settings:
\begin{eqnarray*}
a_1 = 1\\
b_1 = 1\\
k_1 = 2\\
a_2 = 1\\
b_2 = 2\\
k_2 = 1
\end{eqnarray*}
the following conditions are true:
\begin{eqnarray*}
a_1 + b_1 &=& k_1\\
a_2 + k_2 &=& b_2
\end{eqnarray*}
which means we set the terms that resolve to zero:
\begin{eqnarray*}
a_1 + b_1 - k_1\\
a_2 - b_2 + k_2
\end{eqnarray*}
equal to $\pi$:
\begin{equation}
\begin{split}
&\int_{-\pi}^\pi\int_{-\pi}^\pi \Left(1,1,1,2,2,1) dx \; dy = \\
&\frac{1}{4}\left( 
\frac{\sin\left((a_2 - b_2 - k_2)\pi\right)}{a_2 - b_2 - k_2}
+ \frac{\sin\left((a_2 + b_2 - k_2)\pi\right)}{a_2 + b_2 - k_2}
- \pi
- \frac{\sin\left((a_2 + b_2 + k_2)\pi\right)}{a_2 + b_2 + k_2}
\right) \cdot\\
&\left( 
-\frac{\sin\left((a_1 - b_1 - k_1)\pi\right)}{a_1 - b_1 - k_1}
+ \pi
+ \frac{\sin\left((a_1 - b_1 + k_1)\pi\right)}{a_1 - b_1 + k_1}
- \frac{\sin\left((a_1 + b_1 + k_1)\pi\right)}{a_1 + b_1 + k_1}
\right)
\end{split}
\end{equation}
The other sine terms then resolve to zero:
\begin{equation}
\int_{-\pi}^\pi\int_{-\pi}^\pi \Left(1,1,1,2,2,1) dx \; dy =
\frac{1}{4}\left( - \pi \right) \left( + \pi\right) = -\frac{\pi^2}{4}
\end{equation}

The double integral for the right follows similarly, but has slightly different signs:
\begin{equation}
\begin{split}
&\int_{-\pi}^\pi\int_{-\pi}^\pi \Right(a_1, a_2, b_1, b_2, k_1, k_2) dx \; dy = \\
&\frac{1}{4}\left( 
-\frac{\sin\left((a_2 - b_2 - k_2)\pi\right)}{a_2 - b_2 - k_2}
+ \frac{\sin\left((a_2 + b_2 - k_2)\pi\right)}{a_2 + b_2 - k_2}
+ \frac{\sin\left((a_2 - b_2 + k_2)\pi\right)}{a_2 - b_2 + k_2}
- \frac{\sin\left((a_2 + b_2 + k_2)\pi\right)}{a_2 + b_2 + k_2}
\right) \cdot\\
&\left( 
 \frac{\sin\left((a_1 - b_1 - k_1)\pi\right)}{a_1 - b_1 - k_1}
+ \frac{\sin\left((a_1 + b_1 - k_1)\pi\right)}{a_1 + b_1 - k_1}
- \frac{\sin\left((a_1 - b_1 + k_1)\pi\right)}{a_1 - b_1 + k_1}
- \frac{\sin\left((a_1 + b_1 + k_1)\pi\right)}{a_1 + b_1 + k_1}
\right)
\end{split}
\end{equation}

\section{Adding back the scalings}

The eigenfunctions in fact contain a scalar on the front:
\begin{eqnarray*}
\Phi(a_1, a_2)_x &=& a_2 \sin(a_1 x) \cos(a_2 y)\\
\Phi(a_1, a_2)_y &=& -a_1 \cos(a_1 x) \sin(a_2 y)\\
\end{eqnarray*}
The vorticity has no such scaling:
\begin{equation*}
\phi(k_1, k_2) = \sin(k_1 x) \sin(k_2 y)
\end{equation*}
The left and right terms then become:
\begin{eqnarray*}
\Left(a_1, a_2, b_1, b_2, k_1, k_2) &=& b_2 a_1\cdot \cos(a_1 x) \sin(a_2 y) \sin(b_1 x) \cos(b_2 y) \sin(k_1 x) \sin(k_2 y)\\
\Right(a_1, a_2, b_1, b_2, k_1, k_2) &=& b_1 a_2 \cdot \cos(b_1 x) \sin(b_2 y) \sin(a_1 x) \cos(a_2 y) \sin(k_1 x) \sin(k_2 y)
\end{eqnarray*}
The non-zero cases of the integrals then resolve to (within a minus sign):
\begin{equation*}
\int_{-\pi}^\pi\int_{-\pi}^\pi \Left(a_1, a_2, b_1, b_2, k_1, k_2) dx \; dy = b_2 a_1 \cdot \frac{\pi^2}{4}
\end{equation*}
\begin{equation*}
\int_{-\pi}^\pi\int_{-\pi}^\pi \Right(a_1, a_2, b_1, b_2, k_1, k_2) dx \; dy = b_1 a_2 \cdot \frac{\pi^2}{4}
\end{equation*}
After normalizing off the $\frac{\pi^2}{4}$ term, and assuming that all the minuses do not cause the term to cancel off, the final structure coefficient is then:
\begin{equation*}
s(a_1, a_2, b_1, b_2, k_1, k_2) = \frac{b_2 a_1 - b_1 a_2}{b_1^2 + b_2^2}
\end{equation*}
This does seem to align with the terms from Appendix A of the DeWitt paper, though within a factor of 4. Presumably they added a 4 somewhere that we stripped off.

{\bf Note:} After implementation, it looks like the factor of 4 should actually be there. It perhaps arises because the $\pi$ factor gets repeated 4 times when it is double integrated over the $\int_\pi^{-\pi}$ intervals.
\end{document}