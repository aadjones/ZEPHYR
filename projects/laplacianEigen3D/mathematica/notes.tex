\documentclass[12pt,onecolumn]{article}
\usepackage{amsmath}
\usepackage{epsfig}
\usepackage{fullpage}
\usepackage{xcolor}
\usepackage{amssymb}
\usepackage{fancyhdr}

\newcommand{\JJ}{{\bf J}}
\newcommand{\vv}{{\bf v}}
\newcommand{\uu}{{\bf u}}
\newcommand{\II}{{\bf I}}
\newcommand{\BB}{{\bf B}}
\newcommand{\FF}{{\bf F}}
\newcommand{\RR}{{\bf R}}
\newcommand{\UU}{{\bf U}}
\newcommand{\qq}{{\bf q}}
\newcommand{\pp}{{\bf p}}
\newcommand{\VV}{{\bf V}}
\newcommand{\DD}{{\bf D}}
\newcommand{\TT}{{\bf T}}
\newcommand{\trans}{{\bf t}}
\newcommand{\ff}{{\bf f}}
\newcommand{\xx}{{\bf x}}
\newcommand{\boldt}{{\bf t}}
\newcommand{\bolda}{{\bf a}}
\newcommand{\boldb}{{\bf b}}
\newcommand{\bb}{{\bf b}}
\newcommand{\fancyS}{{\mathcal S}}
\newcommand{\boldS}{{\bf S}}
\newcommand{\boldA}{{\bf A}}
\newcommand{\rr}{{\bf r}}
\newcommand{\cc}{{\bf c}}
\newcommand{\dd}{{\bf d}}
\newcommand{\QQ}{{\bf Q}}
\newcommand{\MM}{{\bf M}}
\newcommand{\HH}{\bf H}
\newcommand{\boldB}{{\bf B}}
\newcommand{\CC}{{\bf C}}
\newcommand{\Top}{{\mathbb T}}
\newcommand{\Bottom}{{\mathbb B}}
\newcommand{\functionf}{{\mathbb F}}
\newcommand{\functionr}{{\mathbb R}}
\newcommand{\functions}{{\mathbb S}}
\newcommand{\DB}{\mathbb{D}\Bottom}
\DeclareMathOperator{\sech}{sech}
\DeclareMathOperator{\curl}{curl}
\DeclareMathOperator{\Left}{left}
\DeclareMathOperator{\Right}{right}

\fancyhf{}
\lfoot{ \fancyplain{}{Page \thepage} }
\rfoot{ \fancyplain{}{\today} }
\pagestyle{fancyplain}

\begin{document}
\vspace*{1em}
\section{Working Notes on Laplacian Eigenfunctions}
%Revision: \today
\subsection{Efficiently Computing the 3D Structure Coefficients}
In 3D, all the eigenfunctions, regardless of the velocity direction, have the same form, but the coefficients differ. For simplicity, we can abstract these away into constants:
\begin{eqnarray*}
a_x &=& \alpha_1 \sin(a_1 x) \cos(a_2 y) \cos(a_3 z) \\
a_y &=& \alpha_2 \cos(a_1 x) \sin(a_2 y) \cos(a_3 z) \\
a_z &=& \alpha_3 \cos(a_1 x) \cos(a_2 y) \sin(a_3 z)
\end{eqnarray*}
\begin{eqnarray*}
b_x &=& \beta_1 \sin(b_1 x) \cos(b_2 y) \cos(b_3 z) \\
b_y &=& \beta_2 \cos(b_1 x) \sin(b_2 y) \cos(b_3 z) \\
b_z &=& \beta_3 \cos(b_1 x) \cos(b_2 y) \sin(b_3 z)
\end{eqnarray*}
For example, the $a$ field can be used to generate an $x$ velocity field by setting:
\begin{eqnarray*}
\alpha_1 &=& -(a_2  a_2 + a_3 a_3) \\
\alpha_2 &=& a_1 a_2 \\
\alpha_3 &=& a_1 a_3
\end{eqnarray*}
The vorticity functions flip the trig functions,
\begin{eqnarray*}
v_x &=& \kappa_1 \cos(k_1 x) \sin(k_2 y) \sin(k_3 z) \\
v_y &=& \kappa_2 \sin(k_1 x) \cos(k_2 y) \sin(k_3 z) \\
v_z &=& \kappa_3 \sin(k_1 x) \sin(k_2 y) \cos(k_3 z)
\end{eqnarray*}
and in all cases, one of the $\kappa_*$ terms will be zero, canceling out entire terms later on. 

The conditions for a structure coefficient to be non-zero are then the same as in the 2D case. For each $(a_i, b_i, k_i)$ triplet, two of the terms must sum to the third:
\begin{eqnarray*}
a_i + b_i &=& k_i \\
a_i + k_i &=& b_i \\
b_i + k_i &=& a_i
\end{eqnarray*}
The structure coefficient is then:
\begin{equation}
\begin{split}
-\frac{1}{64} \alpha_3 \beta_2 \kappa_1 
   \left[\frac{\sin((a_1 - b_1 - k_1) \pi)}{a_1 - b_1 - k_1} + 
    \frac{\sin((a_1 + b_1 - k_1) \pi)}{a_1 + b_1 - k_1} + 
    \frac{\sin((a_1 - b_1 + k_1) \pi)}{a_1 - b_1 + k_1} + 
    \frac{\sin((a_1 + b_1 + k_1) \pi)}{a_1 + b_1 + k_1}\right]\\
   \left[-\frac{\sin((a_2 - b_2 - k_2) \pi)}{a_2 - b_2 - k_2)} + 
    \frac{\sin((a_2 + b_2 - k_2) \pi)}{a_2 + b_2 - k_2} + 
    \frac{\sin((a_2 - b_2 + k_2) \pi)}{a_2 - b_2 + k_2} - 
    \frac{\sin((a_2 + b_2 + k_2) \pi)}{a_2 + b_2 + k_2}\right]\\
   \left[\frac{\sin((a_3 - b_3 - k_3) \pi)}{a_3 - b_3 - k_3} + 
    \frac{\sin((a_3 + b_3 - k_3) \pi)}{a_3 + b_3 - k_3} - 
    \frac{\sin((a_3 - b_3 + k_3) \pi)}{a_3 - b_3 + k_3} - 
    \frac{\sin((a_3 + b_3 + k_3) \pi)}{a_3 + b_3 + k_3}\right] +\\ 
 \frac{1}{64} \alpha_3 \beta_1 \kappa_2 \left[-\frac{\sin((a_1 - b_1 - k_1) \pi)}{
     a_1 - b_1 - k_1} + \frac{\sin((a_1 + b_1 - k_1) \pi)}{a_1 + b_1 - k_1} + 
    \frac{\sin((a_1 - b_1 + k_1) \pi)}{a_1 - b_1 + k_1} - 
    \frac{\sin((a_1 + b_1 + k_1) \pi)}{a_1 + b_1 + k_1}\right]\\ 
   \left[\frac{\sin((a_2 - b_2 - k_2) \pi)}{a_2 - b_2 - k_2} + 
    \frac{\sin((a_2 + b_2 - k_2) \pi)}{a_2 + b_2 - k_2} + 
    \frac{\sin((a_2 - b_2 + k_2) \pi)}{a_2 - b_2 + k_2} + 
    \frac{\sin((a_2 + b_2 + k_2) \pi)}{a_2 + b_2 + k_2}\right]\\
   \left[\frac{\sin((a_3 - b_3 - k_3) \pi)}{a_3 - b_3 - k_3} + 
    \frac{\sin((a_3 + b_3 - k_3) \pi)}{a_3 + b_3 - k_3} - 
    \frac{\sin((a_3 - b_3 + k_3) \pi)}{a_3 - b_3 + k_3} - 
    \frac{\sin((a_3 + b_3 + k_3) \pi)}{a_3 + b_3 + k_3}\right] +\\
\frac{1}{64} \alpha_2 \beta_2 \kappa_1 \left[\frac{\sin((a_1 - b_1 - k_1) \pi)}{a_1 - b_1 - k_1} + 
    \frac{\sin((a_1 + b_1 - k_1) \pi)}{a_1 + b_1 - k_1} + 
    \frac{\sin((a_1 - b_1 + k_1) \pi)}{a_1 - b_1 + k_1} + 
    \frac{\sin((a_1 + b_1 + k_1) \pi)}{a_1 + b_1 + k_1}\right]\\
   \left[\frac{\sin((a_2 - b_2 - k_2) \pi)}{a_2 - b_2 - k_2} + 
    \frac{\sin((a_2 + b_2 - k_2) \pi)}{a_2 + b_2 - k_2} - 
    \frac{\sin((a_2 - b_2 + k_2) \pi)}{a_2 - b_2 + k_2} - 
    \frac{\sin((a_2 + b_2 + k_2) \pi)}{a_2 + b_2 + k_2}\right]\\
   \left[-\frac{\sin((a_3 - b_3 - k_3) \pi)}{a_3 - b_3 - k_3} + 
    \frac{\sin((a_3 + b_3 - k_3) \pi)}{a_3 + b_3 - k_3} + 
    \frac{\sin((a_3 - b_3 + k_3) \pi)}{a_3 - b_3 + k_3} - 
    \frac{\sin((a_3 + b_3 + k_3) \pi)}{a_3 + b_3 + k_3}\right] - \\
 \frac{1}{64} \alpha_1 \beta_2 \kappa_2 \left[\frac{\sin((a_1 - b_1 - k_1) \pi)}{a_1 - b_1 - k_1} + 
    \frac{\sin((a_1 + b_1 - k_1) \pi)}{a_1 + b_1 - k_1} - 
    \frac{\sin((a_1 - b_1 + k_1) \pi)}{a_1 - b_1 + k_1} - 
    \frac{\sin((a_1 + b_1 + k_1) \pi)}{a_1 + b_1 + k_1}\right]\\
   \left[\frac{\sin((a_2 - b_2 - k_2) \pi)}{a_2 - b_2 - k_2} + 
    \frac{\sin((a_2 + b_2 - k_2) \pi)}{a_2 + b_2 - k_2} + 
    \frac{\sin((a_2 - b_2 + k_2) \pi)}{a_2 - b_2 + k_2} + 
    \frac{\sin((a_2 + b_2 + k_2) \pi)}{a_2 + b_2 + k_2}\right]\\
   \left[-\frac{\sin((a_3 - b_3 - k_3) \pi)}{a_3 - b_3 - k_3} + 
    \frac{\sin((a_3 + b_3 - k_3) \pi)}{a_3 + b_3 - k_3} + 
    \frac{\sin((a_3 - b_3 + k_3) \pi)}{a_3 - b_3 + k_3} - 
    \frac{\sin((a_3 + b_3 + k_3) \pi)}{a_3 + b_3 + k_3}\right] -\\
 \frac{1}{64} \alpha_2 \beta_1 \kappa_3 \left[-\frac{\sin((a_1 - b_1 - k_1) \pi)}{
     a_1 - b_1 - k_1} + \frac{\sin((a_1 + b_1 - k_1) \pi)}{a_1 + b_1 - k_1} + 
    \frac{\sin((a_1 - b_1 + k_1) \pi)}{a_1 - b_1 + k_1} - 
    \frac{\sin((a_1 + b_1 + k_1) \pi)}{a_1 + b_1 + k_1}\right]\\
   \left[\frac{\sin((a_2 - b_2 - k_2) \pi)}{a_2 - b_2 - k_2} + 
    \frac{\sin((a_2 + b_2 - k_2) \pi)}{a_2 + b_2 - k_2} - 
    \frac{\sin((a_2 - b_2 + k_2) \pi)}{a_2 - b_2 + k_2} - 
    \frac{\sin((a_2 + b_2 + k_2) \pi)}{a_2 + b_2 + k_2}\right]\\
   \left[\frac{\sin((a_3 - b_3 - k_3) \pi)}{a_3 - b_3 - k_3} + 
    \frac{\sin((a_3 + b_3 - k_3) \pi)}{a_3 + b_3 - k_3} + 
    \frac{\sin((a_3 - b_3 + k_3) \pi)}{a_3 - b_3 + k_3} + 
    \frac{\sin((a_3 + b_3 + k_3) \pi)}{a_3 + b_3 + k_3}\right] + \\
 \frac{1}{64} \alpha_1 \beta_2 \kappa_3 \left[\frac{\sin((a_1 - b_1 - k_1) \pi)}{a_1 - b_1 - k_1} + 
    \frac{\sin((a_1 + b_1 - k_1) \pi)}{a_1 + b_1 - k_1} - 
    \frac{\sin((a_1 - b_1 + k_1) \pi)}{a_1 - b_1 + k_1} - 
    \frac{\sin((a_1 + b_1 + k_1) \pi)}{a_1 + b_1 + k_1}\right]\\
   \left[-\frac{\sin((a_2 - b_2 - k_2) \pi)}{a_2 - b_2 - k_2} + 
    \frac{\sin((a_2 + b_2 - k_2) \pi)}{a_2 + b_2 - k_2} + 
    \frac{\sin((a_2 - b_2 + k_2) \pi)}{a_2 - b_2 + k_2} - 
    \frac{\sin((a_2 + b_2 + k_2) \pi)}{a_2 + b_2 + k_2}\right]\\
   \left[\frac{\sin((a_3 - b_3 - k_3) \pi)}{a_3 - b_3 - k_3} + 
    \frac{\sin((a_3 + b_3 - k_3) \pi)}{a_3 + b_3 - k_3} + 
    \frac{\sin((a_3 - b_3 + k_3) \pi)}{a_3 - b_3 + k_3} + 
    \frac{\sin((a_3 + b_3 + k_3) \pi)}{a_3 + b_3 + k_3}\right]
\end{split}
\end{equation}
This can be further simplified by introducing the conditions:
\begin{eqnarray*}
  c_{100}(a,b,k)&=&\begin{cases}
    \pi, & \text{if $a - b - k = 0$}.\\
    0, & \text{otherwise}.
  \end{cases} \\
  c_{110}(a,b,k)&=&\begin{cases}
    \pi, & \text{if $a + b - k = 0$}.\\
    0, & \text{otherwise}.
  \end{cases} \\
  c_{101}(a,b,k)&=&\begin{cases}
    \pi, & \text{if $a - b + k = 0$}.\\
    0, & \text{otherwise}.
  \end{cases} \\
\end{eqnarray*}
Note that the case where the $a + b + k = 0$ does not need to be covered, because the variables are always non-negative, so this case is only true when $a  = b = k = 0$. The expression then simplifies to:
\begin{equation}
\begin{split}
-\frac{1}{64} \alpha_3 \beta_2 \kappa_1 
   \left[c_{100}(a_1, b_1, k_1) + 
    c_{110}(a_1, b_1, k_1) + 
    c_{101}(a_1, b_1, k_1)\right]\\
   \left[-c_{100}(a_2, b_2, k_2) + 
    c_{110}(a_2, b_2, k_2) + 
    c_{101}(a_2, b_2, k_2)\right]\\
   \left[c_{100}(a_3, b_3, k_3) + 
    c_{110}(a_3, b_3, k_3) - 
    c_{101}(a_3, b_3, k_3)\right] \\ 
+ \frac{1}{64} \alpha_3 \beta_1 \kappa_2 \left[-c_{100}(a_1, b_1, k_1) + c_{110}(a_1, b_1, k_1) + 
    c_{101}(a_1, b_1, k_1)\right]\\ 
   \left[c_{100}(a_2, b_2, k_2) + 
    c_{110}(a_2, b_2, k_2) + 
    c_{101}(a_2, b_2, k_2)\right]\\
   \left[c_{100}(a_3, b_3, k_3) + 
    c_{110}(a_3, b_3, k_3) - 
    c_{101}(a_3, b_3, k_3)\right] \\
+\frac{1}{64} \alpha_2 \beta_2 \kappa_1 \left[c_{100}(a_1, b_1, k_1) + 
    c_{110}(a_1, b_1, k_1) + 
    c_{101}(a_1, b_1, k_1)\right]\\
   \left[c_{100}(a_2, b_2, k_2) + 
    c_{110}(a_2, b_2, k_2) - 
    c_{101}(a_2, b_2, k_2)\right]\\
   \left[-c_{100}(a_3, b_3, k_3) + 
    c_{110}(a_3, b_3, k_3) + 
    c_{101}(a_3, b_3, k_3)\right] \\
- \frac{1}{64} \alpha_1 \beta_2 \kappa_2 \left[c_{100}(a_1, b_1, k_1) + 
    c_{110}(a_1, b_1, k_1) - 
    c_{101}(a_1, b_1, k_1)\right]\\
   \left[c_{100}(a_2, b_2, k_2) + 
    c_{110}(a_2, b_2, k_2) + 
    c_{101}(a_2, b_2, k_2)\right]\\
   \left[-c_{100}(a_3, b_3, k_3) + 
    c_{110}(a_3, b_3, k_3) + 
    c_{101}(a_3, b_3, k_3)\right] \\
- \frac{1}{64} \alpha_2 \beta_1 \kappa_3 \left[-c_{100}(a_1, b_1, k_1) + 
 c_{110}(a_1, b_1, k_1) + 
    c_{101}(a_1, b_1, k_1)\right]\\
   \left[c_{100}(a_2, b_2, k_2) + 
    c_{110}(a_2, b_2, k_2) - 
    c_{101}(a_2, b_2, k_2)\right]\\
   \left[c_{100}(a_3, b_3, k_3) + 
    c_{110}(a_3, b_3, k_3) + 
    c_{101}(a_3, b_3, k_3)\right] \\
+ \frac{1}{64} \alpha_1 \beta_2 \kappa_3 \left[c_{100}(a_1, b_1, k_1) + 
    c_{110}(a_1, b_1, k_1) - 
    c_{101}(a_1, b_1, k_1)\right]\\
   \left[-c_{100}(a_2, b_2, k_2) + 
    c_{110}(a_2, b_2, k_2) + 
    c_{101}(a_2, b_2, k_2)\right]\\
   \left[c_{100}(a_3, b_3, k_3) + 
    c_{110}(a_3, b_3, k_3) + 
    c_{101}(a_3, b_3, k_3)\right]
\end{split}
\end{equation}
To visualize the pattern of positives and negatives:
\begin{equation}
\begin{split}
-\frac{1}{64} \alpha_3 \beta_2 \kappa_1 
\left[ \begin{array}{ccc}
1 & 1 & 1 \\
-1 & 1 & 1 \\
1 & 1 & -1 \end{array} \right] 
+ \frac{1}{64} \alpha_3 \beta_1 \kappa_2
\left[ \begin{array}{ccc}
-1 & 1 & 1 \\
1 & 1 & 1 \\
1 & 1 & -1 \end{array} \right] 
+\frac{1}{64} \alpha_2 \beta_2 \kappa_1 
\left[ \begin{array}{ccc}
1 & 1 & 1 \\
1 & 1 & -1 \\
-1 & 1 & 1 \end{array} \right] \\
- \frac{1}{64} \alpha_1 \beta_2 \kappa_2 
\left[ \begin{array}{ccc}
1 & 1 & -1 \\
1 & 1 & 1 \\
-1 & 1 & 1 \end{array} \right]
- \frac{1}{64} \alpha_2 \beta_1 \kappa_3 
\left[ \begin{array}{ccc}
-1 & 1 & 1 \\
1 & 1 & -1 \\
1 & 1 & 1 \end{array} \right]
+ \frac{1}{64} \alpha_1 \beta_2 \kappa_3 
\left[ \begin{array}{ccc}
1 & 1 & -1 \\
-1 & 1 & -1 \\
1 & 1 & 1 \end{array} \right]
\end{split}
\end{equation}

\subsection{Eigenfunction and Vorticity Constants}

An $x$ velocity field:
\begin{eqnarray*}
\alpha_1 &=& -(a_2^2 + a_3^2) \\
\alpha_2 &=& a_1 a_2 \\
\alpha_3 &=& a_1 a_3
\end{eqnarray*}
A $y$ velocity field:
\begin{eqnarray*}
\alpha_1 &=& -a_1 a _2 \\
\alpha_2 &=& a_1^2+ a_3^2\\
\alpha_3 &=& -a_2 a_3
\end{eqnarray*}
A $z$ velocity field:
\begin{eqnarray*}
\alpha_1 &=& a_1 a _3 \\
\alpha_2 &=& a_2 a_3\\
\alpha_3 &=& -(a_1^2 + a_2^2)
\end{eqnarray*}
An $x$ vorticity field:
\begin{eqnarray*}
\kappa_1 &=& 0 \\
\kappa_2 &=& k_3\\
\kappa_3 &=& -k_2
\end{eqnarray*}
A $y$ vorticity field:
\begin{eqnarray*}
\kappa_1 &=& k_3 \\
\kappa_2 &=& 0\\
\kappa_3 &=& -k_1
\end{eqnarray*}
A $z$ vorticity field:
\begin{eqnarray*}
\kappa_1 &=& k_2 \\
\kappa_2 &=& -k_1\\
\kappa_3 &=& 0
\end{eqnarray*}

\subsection{Post-Implementation}

Depending on the Mathematica run, we seem to get slightly different terms in the integral. Combining them seems to yield:
\begin{equation}
\begin{split}
-\frac{1}{64} \alpha_3 \beta_2 \kappa_1 
   \left[c_{100}(a_1, b_1, k_1) + 
    c_{110}(a_1, b_1, k_1) + 
    c_{101}(a_1, b_1, k_1)\right]\\
   \left[-c_{100}(a_2, b_2, k_2) + 
    c_{110}(a_2, b_2, k_2) + 
    c_{101}(a_2, b_2, k_2)\right]\\
   \left[c_{100}(a_3, b_3, k_3) + 
    c_{110}(a_3, b_3, k_3) - 
    c_{101}(a_3, b_3, k_3)\right] \\ 
+ \frac{1}{64} \alpha_3 \beta_1 \kappa_2 \left[-c_{100}(a_1, b_1, k_1) + c_{110}(a_1, b_1, k_1) + 
    c_{101}(a_1, b_1, k_1)\right]\\ 
   \left[c_{100}(a_2, b_2, k_2) + 
    c_{110}(a_2, b_2, k_2) + 
    c_{101}(a_2, b_2, k_2)\right]\\
   \left[c_{100}(a_3, b_3, k_3) + 
    c_{110}(a_3, b_3, k_3) - 
    c_{101}(a_3, b_3, k_3)\right] \\
+\frac{1}{64} (\alpha_2 \beta_2 \kappa_1 + \alpha_2 \beta_3 \kappa_1) \left[c_{100}(a_1, b_1, k_1) + 
    c_{110}(a_1, b_1, k_1) + 
    c_{101}(a_1, b_1, k_1)\right]\\
   \left[c_{100}(a_2, b_2, k_2) + 
    c_{110}(a_2, b_2, k_2) - 
    c_{101}(a_2, b_2, k_2)\right]\\
   \left[-c_{100}(a_3, b_3, k_3) + 
    c_{110}(a_3, b_3, k_3) + 
    c_{101}(a_3, b_3, k_3)\right] \\
- \frac{1}{64} (\alpha_1 \beta_2 \kappa_2 + \alpha_1 \beta_3 \kappa_2) \left[c_{100}(a_1, b_1, k_1) + 
    c_{110}(a_1, b_1, k_1) - 
    c_{101}(a_1, b_1, k_1)\right]\\
   \left[c_{100}(a_2, b_2, k_2) + 
    c_{110}(a_2, b_2, k_2) + 
    c_{101}(a_2, b_2, k_2)\right]\\
   \left[-c_{100}(a_3, b_3, k_3) + 
    c_{110}(a_3, b_3, k_3) + 
    c_{101}(a_3, b_3, k_3)\right] \\
- \frac{1}{64} \alpha_2 \beta_1 \kappa_3 \left[-c_{100}(a_1, b_1, k_1) + 
 c_{110}(a_1, b_1, k_1) + 
    c_{101}(a_1, b_1, k_1)\right]\\
   \left[c_{100}(a_2, b_2, k_2) + 
    c_{110}(a_2, b_2, k_2) - 
    c_{101}(a_2, b_2, k_2)\right]\\
   \left[c_{100}(a_3, b_3, k_3) + 
    c_{110}(a_3, b_3, k_3) + 
    c_{101}(a_3, b_3, k_3)\right] \\
+ \frac{1}{64} \alpha_1 \beta_2 \kappa_3 \left[c_{100}(a_1, b_1, k_1) + 
    c_{110}(a_1, b_1, k_1) - 
    c_{101}(a_1, b_1, k_1)\right]\\
   \left[-c_{100}(a_2, b_2, k_2) + 
    c_{110}(a_2, b_2, k_2) + 
    c_{101}(a_2, b_2, k_2)\right]\\
   \left[c_{100}(a_3, b_3, k_3) + 
    c_{110}(a_3, b_3, k_3) + 
    c_{101}(a_3, b_3, k_3)\right]
\end{split}
\end{equation}
This does not seem to work. The $(\alpha_2 \beta_2 \kappa_1 + \alpha_2 \beta_3 \kappa_1)$ can cancel, yielding a spurious zero entry. Switching over to {\em only} the higher index on $\beta$ seems to work (why?). The symmetry of the equation at least looks a little nicer now; the $(\alpha, \beta, \kappa)$ coefficient is always a permutation on $(1,2,3)$.
\begin{equation}
\begin{split}
-\frac{1}{64} \alpha_3 \beta_2 \kappa_1 
   \left[c_{100}(a_1, b_1, k_1) + 
    c_{110}(a_1, b_1, k_1) + 
    c_{101}(a_1, b_1, k_1)\right]\\
   \left[-c_{100}(a_2, b_2, k_2) + 
    c_{110}(a_2, b_2, k_2) + 
    c_{101}(a_2, b_2, k_2)\right]\\
   \left[c_{100}(a_3, b_3, k_3) + 
    c_{110}(a_3, b_3, k_3) - 
    c_{101}(a_3, b_3, k_3)\right] \\ 
+ \frac{1}{64} \alpha_3 \beta_1 \kappa_2 \left[-c_{100}(a_1, b_1, k_1) + c_{110}(a_1, b_1, k_1) + 
    c_{101}(a_1, b_1, k_1)\right]\\ 
   \left[c_{100}(a_2, b_2, k_2) + 
    c_{110}(a_2, b_2, k_2) + 
    c_{101}(a_2, b_2, k_2)\right]\\
   \left[c_{100}(a_3, b_3, k_3) + 
    c_{110}(a_3, b_3, k_3) - 
    c_{101}(a_3, b_3, k_3)\right] \\
+\frac{1}{64} \alpha_2 \beta_3 \kappa_1 \left[c_{100}(a_1, b_1, k_1) + 
    c_{110}(a_1, b_1, k_1) + 
    c_{101}(a_1, b_1, k_1)\right]\\
   \left[c_{100}(a_2, b_2, k_2) + 
    c_{110}(a_2, b_2, k_2) - 
    c_{101}(a_2, b_2, k_2)\right]\\
   \left[-c_{100}(a_3, b_3, k_3) + 
    c_{110}(a_3, b_3, k_3) + 
    c_{101}(a_3, b_3, k_3)\right] \\
- \frac{1}{64} \alpha_1 \beta_3 \kappa_2 \left[c_{100}(a_1, b_1, k_1) + 
    c_{110}(a_1, b_1, k_1) - 
    c_{101}(a_1, b_1, k_1)\right]\\
   \left[c_{100}(a_2, b_2, k_2) + 
    c_{110}(a_2, b_2, k_2) + 
    c_{101}(a_2, b_2, k_2)\right]\\
   \left[-c_{100}(a_3, b_3, k_3) + 
    c_{110}(a_3, b_3, k_3) + 
    c_{101}(a_3, b_3, k_3)\right] \\
- \frac{1}{64} \alpha_2 \beta_1 \kappa_3 \left[-c_{100}(a_1, b_1, k_1) + 
 c_{110}(a_1, b_1, k_1) + 
    c_{101}(a_1, b_1, k_1)\right]\\
   \left[c_{100}(a_2, b_2, k_2) + 
    c_{110}(a_2, b_2, k_2) - 
    c_{101}(a_2, b_2, k_2)\right]\\
   \left[c_{100}(a_3, b_3, k_3) + 
    c_{110}(a_3, b_3, k_3) + 
    c_{101}(a_3, b_3, k_3)\right] \\
+ \frac{1}{64} \alpha_1 \beta_2 \kappa_3 \left[c_{100}(a_1, b_1, k_1) + 
    c_{110}(a_1, b_1, k_1) - 
    c_{101}(a_1, b_1, k_1)\right]\\
   \left[-c_{100}(a_2, b_2, k_2) + 
    c_{110}(a_2, b_2, k_2) + 
    c_{101}(a_2, b_2, k_2)\right]\\
   \left[c_{100}(a_3, b_3, k_3) + 
    c_{110}(a_3, b_3, k_3) + 
    c_{101}(a_3, b_3, k_3)\right]
\end{split}
\end{equation}

\subsection{Zero Modes}

Recalling that eigenfunctions are of the form:
\begin{eqnarray*}
a_x &=& \alpha_1 \sin(a_1 x) \cos(a_2 y) \cos(a_3 z) \\
a_y &=& \alpha_2 \cos(a_1 x) \sin(a_2 y) \cos(a_3 z) \\
a_z &=& \alpha_3 \cos(a_1 x) \cos(a_2 y) \sin(a_3 z)
\end{eqnarray*}
Eigenfunctions with zero amplitude also appear, and these should be discarded for efficiency. For a specific frequency triplet, $(k_1, k_2, k_3)$, and a component $i$, which can be equal to $x$, $y$, or $z$, two conditions will zero out the entire eigenfunction.

If two of the $k_*$ frequencies are zero, the entire function goes to zero, because the $\alpha_1$, $\alpha_2$, and $\alpha_3$ will all be forced to zero. If one of the $k_j$ entries is zero, and $j = i$, this will also force the eigenfunction to zero. In this case, the $k_j$ will force two of the $\alpha_*$ entries to zero, and the $k_j$ itself will force the $\sin(k_j *)$ to zero, effectively canceling out the third component as well.

\subsection{The Fourier Delta Functions}

Each eigenfunction generates a delta function in Fourier space (really cosine-cosine-sine space). For a given triplet $(k_1, k_2, k_3)$, the delta function then appears at, for each component,
\begin{eqnarray*}
x &=& (k_1 - 1, k_2, k_3) \\
y &=& (k_1, k_2 - 1, k_3) \\
z &=& (k_1, k_2, k_3 - 1) \\
\end{eqnarray*}
The minus one appears when a component has DST applied instead of DCT, since there is no DC component in DST, so everything shifts over by one.



\end{document}